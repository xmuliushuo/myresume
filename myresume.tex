%% start of file `template-zh.tex'.
%% Copyright 2006-2013 Xavier Danaux (xdanaux@gmail.com).
%
% This work may be distributed and/or modified under the
% conditions of the LaTeX Project Public License version 1.3c,
% available at http://www.latex-project.org/lppl/.


\documentclass[11pt,a4paper,sans]{moderncv}   % possible options include font size ('10pt', '11pt' and '12pt'), paper size ('a4paper', 'letterpaper', 'a5paper', 'legalpaper', 'executivepaper' and 'landscape') and font family ('sans' and 'roman')

% moderncv 主题
\moderncvstyle{classic}                        % 选项参数是 ‘casual’, ‘classic’, ‘oldstyle’ 和 ’banking’
\moderncvcolor{blue}                          % 选项参数是 ‘blue’ (默认)、‘orange’、‘green’、‘red’、‘purple’ 和 ‘grey’
%\nopagenumbers{}                             % 消除注释以取消自动页码生成功能

% 字符编码
\usepackage[utf8]{inputenc}                   % 替换你正在使用的编码
\usepackage{CJKutf8}

% 调整页面出血
\usepackage[scale=0.9]{geometry}
\setlength{\hintscolumnwidth}{3cm}           % 如果你希望改变日期栏的宽度

% 个人信息
\name{X}{X}
%\title{简历题目 (可选项)}                     % 可选项、如不需要可删除本行
%\address{街道及门牌号}{邮编及城市}            % 可选项、如不需要可删除本行
\phone[mobile]{152xxxxxx}              % 可选项、如不需要可删除本行
%\phone[fixed]{+2~(345)~678~901}               % 可选项、如不需要可删除本行
%\phone[fax]{+3~(456)~789~012}                 % 可选项、如不需要可删除本行
\email{xxxxxxxxx@gmail.com}                    % 可选项、如不需要可删除本行
\homepage{http://xxxxxxxx.com}                  % 可选项、如不需要可删除本行
\homepage{https://github.com/xxxxxx}
\extrainfo{xx大学计算机硕士}                 % 可选项、如不需要可删除本行
\photo[64pt][0.4pt]{picture}                  % ‘64pt’是图片必须压缩至的高度、‘0.4pt‘是图片边框的宽度 (如不需要可调节至0pt)、’picture‘ 是图片文件的名字;可选项、如不需要可删除本行
%\quote{引言(可选项)}                          % 可选项、如不需要可删除本行

% 显示索引号;仅用于在简历中使用了引言
%\makeatletter
%\renewcommand*{\bibliographyitemlabel}{\@biblabel{\arabic{enumiv}}}
%\makeatother

% 分类索引

%\usepackage{multibib}
%\newcites{book,misc}{{Books},{Others}}
%----------------------------------------------------------------------------------
%            内容
%----------------------------------------------------------------------------------
\begin{document}
\begin{CJK}{UTF8}{gbsn}                       % 详情参阅CJK文件包
\maketitle

\section{教育背景}
\cventry{2011.09 -- 至今}{硕士}{xx大学}{计算机科学与技术}{\textit{Top 5\%}}{研究方向:xxxxxxxx。}
\cventry{2007.09 -- 2011.07}{学士}{xx大学}{计算机科学与技术}{\textit{Top 10\%}}{在读期间学习成绩优异,基础知识扎实,保送进入xx大学计算机系攻读硕士学位。}

\section{知识技能}
\cvlistitem{熟悉C/C++,熟悉Linux平台C语言编程,熟悉Linux网络编程。}
\cvlistitem{熟悉Linux内核,尤其熟悉Linux的I/O模块,精通Linux磁盘调度算法。}
\cvlistitem{了解Python、Shell等脚本语言,了解MySQL等关系数据库,了解.Net程序开发。}
\cvlistitem{英语CET-6 504。}

\section{项目经验}
\cventry{2012.09 -- 2013.08}{xxxxx}{xxxxx}{}{}{
\begin{itemize}%
\item 开发xxxxx。
\item 研究xxxxx。
\item 主要技术:Linux C/C++ 网络编程,Linux内核编程。
\end{itemize}}
\cventry{2012.07 -- 2012.09}{实习软件工程师}{xxxx股份有限公司}{}{}{
\begin{itemize}
\item xx云平台的预研。
\item 学习了解开源的云平台管理软件CloudStack以及xenserver、kvm等虚拟化软件,搭建公司的云平台。
\end{itemize}}
\cventry{2011.03 -- 2012.09}{实习软件工程师}{xxxxx信息科技有限公司}{}{}{
\begin{itemize}
\item 负责开发公司的外包项目,包括旅游网站的开发和iPad电子杂志的开发。
\item 主要技术:Python,Objective-C
\end{itemize}}
\cventry{2010.06 -- 2010.12}{xxxxx}{xx大学xx实验室}{}{}{
\begin{itemize}
\item 主要负责数据库表自动生成模块的开发。
\item 主要技术:ASP.NET,SQL Sever数据库
\end{itemize}}

\section{社会实践}
\cventry{2013.02 -- 2013.07}{《操作系统》实验课程助教}{}{}{}{
在Nachos虚拟操作系统上实现锁和信号量等同步原语、利用时间栅栏实现电梯算法以及完善系统调用使其支持多道程序的运行等。}
\cventry{2013.02 -- 2013.07}{《UNIX程序设计》助教}{}{}{}{
UNIX/Linux环境下的C语言编程,包括文件操作,多进程、多线程编程,信号量的使用等。}

\section{获奖情况}
\cvitem{2012.09 -- 2013.07}{xx大学校一等奖学金}
\cvitem{2011.09 -- 2012.07}{xx大学校一等奖学金和xxxx安全知识竞赛二等奖}
\cvitem{2010.09 -- 2011.07}{xx大学校二等奖学金和建设银行奖学金}
\cvitem{2009.09 -- 2010.07}{xx大学校一等奖学金和xx大学校三好学生}
\cvitem{2007.09 -- 2008.07}{xx大学校二等奖学金}

% 来自BibTeX文件但不使用multibib包的出版物
%\renewcommand*{\bibliographyitemlabel}{\@biblabel{\arabic{enumiv}}}% BibTeX的数字标签
\nocite{*}
\bibliographystyle{plain}
\bibliography{publications}                    % 'publications' 是BibTeX文件的文件名

% 来自BibTeX文件并使用multibib包的出版物
%\section{出版物}
%\nocitebook{book1,book2}
%\bibliographystylebook{plain}
%\bibliographybook{publications}               % 'publications' 是BibTeX文件的文件名
%\nocitemisc{misc1,misc2,misc3}
%\bibliographystylemisc{plain}
%\bibliographymisc{publications}               % 'publications' 是BibTeX文件的文件名

\clearpage\end{CJK}
\end{document}


%% 文件结尾 `template-zh.tex'.
